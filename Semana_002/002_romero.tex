%%
%% (
%%  )\ )                             (
%%  (()/(   (            (             )\  )   (
%%   /(_))  ))\   (       ))\  (   (   (()/(   ))\
%%   (_))  /((_)  )\  )  /((_) )\  )\   ((_))/((_)
%%   | _ \(_))(  _(_/( (_) )  ((_)((_)  _| |(_))
%%   |   /| || || ' \))/ -_)/ _|/ _ \/ _` |/ -_)
%%   |_|_\ \_,_||_||_| \___|\__|\___/\__,_|\___|
%%

\documentclass{article}
\usepackage[utf8x]{inputenc}
\usepackage{amsmath}
\usepackage{amsfonts}
\usepackage{amssymb}
\usepackage{graphicx} % Paquete para incluir imágenes en el documento LaTeX
\usepackage{hyperref}
\hypersetup{
  colorlinks=true,
  linkcolor=blue,
  filecolor=magenta,
  urlcolor=cyan,
}
\urlstyle{same}
\usepackage{varwidth}

\newcommand\tab[1][1cm]{\hspace*{#1}}

\usepackage{multirow}

\usepackage[a4paper,rmargin=1.5cm,lmargin=1.5cm,top=1.5cm,bottom=1.5cm]{geometry}

\usepackage{pdfpages}

\usepackage{xcolor}
\usepackage{minted}
\usemintedstyle[cpp]{monokai}
\usemintedstyle[python]{paraiso-dark}
\usemintedstyle[./pseudocode.py:PseudocodeLexer -x]{rainbow_dash}
\definecolor{LightGray}{gray}{0.98}
\definecolor{DarkGray}{gray}{0.1}
\definecolor{MidGray}{gray}{0.8}
\definecolor{codegreen}{rgb}{0,0.6,0}
\definecolor{codegray}{rgb}{0.5,0.5,0.5}
\definecolor{codepurple}{rgb}{0.58,0,0.82}
\definecolor{backcolour}{rgb}{0.95,0.95,0.92}

\setlength{\parindent}{0px}  % Setea la indentacion de la primera linea de cada parrafo a cero pixeles.


\title{Resolución de la primera semana}
\author{@RuneCode}

\begin{document}
%% Portada
\includepdf{./portada/portada.pdf}


%% ####################################################################################
%%    Inicio del Documento
%% ####################################################################################
\section*{Segunda Semana}%
En esta clase se revisaron los temas de:
\begin{itemize}
\item \textbf{Condicional Simple}
\item \textbf{Condicional Doble}
\item \textbf{Condicional anidado}
\item \textbf{Condicional múltiple}
\item \textbf{Uso de operadores lógicos}
\end{itemize}
\vspace{1cm}
\textbf{Nota:} Te recomendamos realizar los ejercicios antes de ver su solución
ya que el objetivo de este documento es presentarte una de las muchas posibles
soluciones y así tengas una modelo con el que puedas comparar. Si tienes alguna
sugerencia de mejora puedes comunicarte por correo a la dirección
\href{mailto:gprunecode@gmail.com}{gprunecode@gmail.com}.

%% ####################################################################################
%%    Condicional Simple
%% ####################################################################################
\section{Condicional Simple}%
Ejecuta las sentencias en el caso que la expresion lógica sea Verdad.

%Pseudocódigo
\inputminted
[
  frame=lines,
  framesep=2mm,
  baselinestretch=1.2,
  rulecolor=\color{black!30},
  %fontsize=\footnotesize,
  bgcolor=LightGray,
  %linenos
]
{./pseudocode.py:PseudocodeLexer -x}
{./pseudocodigo/condicional_simple.algo}


\subsection*{Ejemplo 1}%
Los trabajadores con más de 10 años de antigüedad recibirán un bono de 10\% de
su básico. Hallar bono y total ingresos.\\

%Pseudocódigo
\underline{\textit{Resolución en pseudocódigo}}\\ 
\inputminted
[
  frame=lines,
  framesep=2mm,
  baselinestretch=1.2,
  rulecolor=\color{black!30},
  %fontsize=\footnotesize,
  bgcolor=LightGray,
  %linenos
]
{./pseudocode.py:PseudocodeLexer -x}
{./pseudocodigo/001_ejemplo.algo}
\clearpage

%C++
\underline{\textit{Resolución en C++}}\\
\inputminted
[
  frame=lines,
  framesep=2mm,
  baselinestretch=1.2,
  rulecolor=\color{black!30},
  %fontsize=\footnotesize,
  bgcolor=DarkGray,
  %linenos
]
{cpp}
{./cpp/001_ejemplo.cpp}


%% ####################################################################################
%%    Condicional Doble
%% ####################################################################################
\section{Condicional Doble}%
Ejecuta sentenciaV si expresionLogica es Verda y ejecutará setenciaF si
expresionLogica es Falsa.

%Pseudocódigo
\inputminted
[
  frame=lines,
  framesep=2mm,
  baselinestretch=1.2,
  rulecolor=\color{black!30},
  %fontsize=\footnotesize,
  bgcolor=LightGray,
  %linenos
]
{./pseudocode.py:PseudocodeLexer -x}
{./pseudocodigo/condicional_doble.algo}


\subsection*{Ejemplo 2}%
Los trabajadores con más de 10 años de antigüedad recibirán un bono de 10\% de
su básico, los demás sólo 5\%. Hallar bono y total de ingresos.\\

%Pseudocódigo
\underline{\textit{Resolución en pseudocódigo}}\\ 
\inputminted
[
  frame=lines,
  framesep=2mm,
  baselinestretch=1.2,
  rulecolor=\color{black!30},
  %fontsize=\footnotesize,
  bgcolor=LightGray,
  %linenos
]
{./pseudocode.py:PseudocodeLexer -x}
{./pseudocodigo/002_ejemplo.algo}

%C++
\underline{\textit{Resolución en C++}}\\
\inputminted
[
  frame=lines,
  framesep=2mm,
  baselinestretch=1.2,
  rulecolor=\color{black!30},
  %fontsize=\footnotesize,
  bgcolor=DarkGray,
  %linenos
]
{cpp}
{./cpp/002_ejemplo.cpp}


%% ####################################################################################
%%    Condicional Anidado
%% ####################################################################################
\section{Condicional Anidado}%
Se ubica un condicional simple o doble dentro de otro condicional. Se puede
anidar en la parte entonces o en sino. Se sugiere ubicar el anidamiento en la
sección sino.

%Pseudocódigo
\inputminted
[
  frame=lines,
  framesep=2mm,
  baselinestretch=1.2,
  rulecolor=\color{black!30},
  %fontsize=\footnotesize,
  bgcolor=LightGray,
  %linenos
]
{./pseudocode.py:PseudocodeLexer -x}
{./pseudocodigo/condicional_anidado.algo}


\subsection*{Ejemplo 3}%
La empresa "Sedapal" facturará para consumos inferiores a 100 $m^3$, S/.1 x
$m^3$ para consumos de 100 hasta 500 $m^3$, S/.1.5 x $m^3$ y para consumos
superiores a 500 $m^3$, S/.2 x $m^3$. Hallar importe a pagar.\\

%Pseudocódigo
\underline{\textit{Resolución en pseudocódigo}}\\ 
\inputminted
[
  frame=lines,
  framesep=2mm,
  baselinestretch=1.2,
  rulecolor=\color{black!30},
  %fontsize=\footnotesize,
  bgcolor=LightGray,
  %linenos
]
{./pseudocode.py:PseudocodeLexer -x}
{./pseudocodigo/003_ejemplo.algo}

%C++
\underline{\textit{Resolución en C++}}\\
\inputminted
[
  frame=lines,
  framesep=2mm,
  baselinestretch=1.2,
  rulecolor=\color{black!30},
  %fontsize=\footnotesize,
  bgcolor=DarkGray,
  %linenos
]
{cpp}
{./cpp/003_ejemplo.cpp}


%% ####################################################################################
%%    Condicional Multiple
%% ####################################################################################
\section{Condicional Múltiple}%
Usaremos caso ... vale. Sólo una de las opciones será verdadera y se ejecutarán
las sentencias asociadas a ella. La variable evaluada debe ser entera o
carácter.

%Pseudocódigo
\inputminted
[
  frame=lines,
  framesep=2mm,
  baselinestretch=1.2,
  rulecolor=\color{black!30},
  %fontsize=\footnotesize,
  bgcolor=LightGray,
  %linenos
]
{./pseudocode.py:PseudocodeLexer -x}
{./pseudocodigo/condicional_multiple.algo}

\subsection*{Ejemplo 4}%
Ingresar un entero (de 1 a 7) y mostrar el día correspondiente.\\

%Pseudocódigo
\underline{\textit{Resolución en pseudocódigo}}\\ 
\inputminted
[
  frame=lines,
  framesep=2mm,
  baselinestretch=1.2,
  rulecolor=\color{black!30},
  %fontsize=\footnotesize,
  bgcolor=LightGray,
  %linenos
]
{./pseudocode.py:PseudocodeLexer -x}
{./pseudocodigo/004_ejemplo.algo}

%C++
\underline{\textit{Resolución en C++}}\\
\inputminted
[
  frame=lines,
  framesep=2mm,
  baselinestretch=1.2,
  rulecolor=\color{black!30},
  %fontsize=\footnotesize,
  bgcolor=DarkGray,
  %linenos
]
{cpp}
{./cpp/004_ejemplo.cpp}


%% ####################################################################################
%%    Operadores logicos
%% ####################################################################################
\section{Uso de operadores lógicos}%
\begin{itemize}
  \item El resultado de una expresión lógica es el valor verdadero o falso.
  \item Operadores relacionales: \textbf{$=$, $<>$, $<$, $<=$, $>$, $>=$}
  \item Operadores Lógicos: \textbf{No, Y, O}
  \item Para plantear expresiones lógicas más complejas se puede usar operadores lógicos.
\end{itemize}

\subsection*{Ejercicio 1}%
Calcular las raíces de una ecuación de segundo grado. Considere las diferentes
situaciones que se pueden dar.

\begin{equation*}
  x = \frac{-b \pm \sqrt{b^2 - 4ac}}{2a}
\end{equation*}

%Pseudocódigo
\textit{Resolución en pseudocódigo}\\ 
\inputminted
[
  frame=lines,
  framesep=2mm,
  baselinestretch=1.2,
  rulecolor=\color{black!30},
  %fontsize=\footnotesize,
  bgcolor=LightGray,
  %linenos
]
{./pseudocode.py:PseudocodeLexer -x}
{./pseudocodigo/001_ejercicio.algo}

%C++
\textit{Resolución en C++}\\
\inputminted
[
  frame=lines,
  framesep=2mm,
  baselinestretch=1.2,
  rulecolor=\color{black!30},
  %fontsize=\footnotesize,
  bgcolor=DarkGray,
  %linenos
]
{cpp}
{./cpp/001_ejercicio.cpp}

\clearpage

\subsection*{Ejercicio 2}%
Calcular el pago por ciclo de un alumno de una Universidad, si se ingresan,
créditos inscritos, categoría, matrícula (1: normal, 2: extemporánea). El pago
por crédito depende de la categoría de acuerdo a la siguiente tabla:

\begin{table}[htbp]
  \begin{center}
    \begin{tabular}{|l|l|}
      \hline
      CATEGORÍA & PAGO CRÉDITO \\
      \hline \hline
      A & 125.00 \\ \hline
      B & 150.00 \\ \hline
      C & 180.00 \\ \hline
    \end{tabular}
  \end{center}
\end{table}
Por matrícula extemporánea se paga un recargo de 40.00 soles.\\

%Pseudocódigo
\textit{Resolución en pseudocódigo}\\ 
\inputminted
[
  frame=lines,
  framesep=2mm,
  baselinestretch=1.2,
  rulecolor=\color{black!30},
  %fontsize=\footnotesize,
  bgcolor=LightGray,
  %linenos
]
{./pseudocode.py:PseudocodeLexer -x}
{./pseudocodigo/002_ejercicio.algo}

%C++
\underline{\textit{Resolución en C++}}\\
\inputminted
[
  frame=lines,
  framesep=2mm,
  baselinestretch=1.2,
  rulecolor=\color{black!30},
  %fontsize=\footnotesize,
  bgcolor=DarkGray,
  %linenos
]
{cpp}
{./cpp/002_ejercicio.cpp}


%% ####################################################################################
%%    Practica N 2
%% ####################################################################################
\section{Practica N 2}%

\subsection*{Ejercicio 1}%
Calcular el pago semanal de un trabajador. Los datos a ingresar son: Total de
horas trabajadas y el pago por hora.\\
Si el total de horas trabajadas es mayor a 40, la diferencia se considera como
horas extras y se paga un 50\% más que una hora normal.\\
Si el sueldo bruto es mayor a S/.500.00, se descuenta un 10\% en caso contrario
el descuento es 0.\\

%Pseudocódigo
\underline{\textit{Resolución en pseudocódigo}}\\ 
\inputminted
[
  frame=lines,
  framesep=2mm,
  baselinestretch=1.2,
  rulecolor=\color{black!30},
  %fontsize=\footnotesize,
  bgcolor=LightGray,
  %linenos
]
{./pseudocode.py:PseudocodeLexer -x}
{./pseudocodigo/001_practica.algo}

%C++
\underline{\textit{Resolución en C++}}\\
\inputminted
[
  frame=lines,
  framesep=2mm,
  baselinestretch=1.2,
  rulecolor=\color{black!30},
  %fontsize=\footnotesize,
  bgcolor=DarkGray,
  %linenos
]
{cpp}
{./cpp/001_practica.cpp}


\subsection*{Ejercicio 2}%
A un trabajador le descuentan de su sueldo el 10\% si su sueldo es menor o
igual a 1000, por encima de 1000 hasta 2000 el 5\% del adicional, y por encima
de 2000 el 3\% del adicional. Calcular el descuento y el sueldo neto que recibe
el trabajador dado un sueldo.\\

%Pseudocódigo
\underline{\textit{Resolución en pseudocódigo}}\\ 
\inputminted
[
  frame=lines,
  framesep=2mm,
  baselinestretch=1.2,
  rulecolor=\color{black!30},
  %fontsize=\footnotesize,
  bgcolor=LightGray,
  %linenos
]
{./pseudocode.py:PseudocodeLexer -x}
{./pseudocodigo/002_practica.algo}

%C++
\underline{\textit{Resolución en C++}}\\
\inputminted
[
  frame=lines,
  framesep=2mm,
  baselinestretch=1.2,
  rulecolor=\color{black!30},
  %fontsize=\footnotesize,
  bgcolor=DarkGray,
  %linenos
]
{cpp}
{./cpp/002_practica.cpp}


\subsection*{Ejercicio 3}%
Ordene de mayor a menor, 3 números ingresados por teclado.\\

%Pseudocódigo
\underline{\textit{Resolución en pseudocódigo}}\\ 
\inputminted
[
  frame=lines,
  framesep=2mm,
  baselinestretch=1.2,
  rulecolor=\color{black!30},
  %fontsize=\footnotesize,
  bgcolor=LightGray,
  %linenos
]
{./pseudocode.py:PseudocodeLexer -x}
{./pseudocodigo/003_practica.algo}

%C++
\underline{\textit{Resolución en C++}}\\
\inputminted
[
  frame=lines,
  framesep=2mm,
  baselinestretch=1.2,
  rulecolor=\color{black!30},
  %fontsize=\footnotesize,
  bgcolor=DarkGray,
  %linenos
]
{cpp}
{./cpp/003_practica.cpp}


\subsection*{Ejercicio 4}%
Dado un tiempo en minutos, calcular los días, horas y minutos que le
corresponden.\\

%Pseudocódigo
\underline{\textit{Resolución en pseudocódigo}}\\ 
\inputminted
[
  frame=lines,
  framesep=2mm,
  baselinestretch=1.2,
  rulecolor=\color{black!30},
  %fontsize=\footnotesize,
  bgcolor=LightGray,
  %linenos
]
{./pseudocode.py:PseudocodeLexer -x}
{./pseudocodigo/004_practica.algo}

%C++
\underline{\textit{Resolución en C++}}\\
\inputminted
[
  frame=lines,
  framesep=2mm,
  baselinestretch=1.2,
  rulecolor=\color{black!30},
  %fontsize=\footnotesize,
  bgcolor=DarkGray,
  %linenos
]
{cpp}
{./cpp/004_practica.cpp}


\subsection*{Ejercicio 5}%
Dado tres datos enteros positivos, que representen las longitudes de un posible
triángulo, determine si los datos corresponden a un triángulo. En caso
afirmativo, escriba si el triángulo es equilátero, isósceles o escaleno.
Calcule además su área.\\

%Pseudocódigo
\underline{\textit{Resolución en pseudocódigo}}\\ 
\inputminted
[
  frame=lines,
  framesep=2mm,
  baselinestretch=1.2,
  rulecolor=\color{black!30},
  %fontsize=\footnotesize,
  bgcolor=LightGray,
  %linenos
]
{./pseudocode.py:PseudocodeLexer -x}
{./pseudocodigo/005_practica.algo}

%C++
\underline{\textit{Resolución en C++}}\\
\inputminted
[
  frame=lines,
  framesep=2mm,
  baselinestretch=1.2,
  rulecolor=\color{black!30},
  %fontsize=\footnotesize,
  bgcolor=DarkGray,
  %linenos
]
{cpp}
{./cpp/005_practica.cpp}



\subsection*{Ejercicio 6}%
Dada la hora del día en horas, minutos y segundos encuentre la hora del
siguiente segundo.\\

%Pseudocódigo
\underline{\textit{Resolución en pseudocódigo}}\\ 
\inputminted
[
  frame=lines,
  framesep=2mm,
  baselinestretch=1.2,
  rulecolor=\color{black!30},
  %fontsize=\footnotesize,
  bgcolor=LightGray,
  %linenos
]
{./pseudocode.py:PseudocodeLexer -x}
{./pseudocodigo/006_practica.algo}

%C++
\underline{\textit{Resolución en C++}}\\
\inputminted
[
  frame=lines,
  framesep=2mm,
  baselinestretch=1.2,
  rulecolor=\color{black!30},
  %fontsize=\footnotesize,
  bgcolor=DarkGray,
  %linenos
]
{cpp}
{./cpp/006_practica.cpp}


\subsection*{Ejercicio 7}%
Una compañía de alquiler de autos emite la factura de sus clientes teniendo en
cuenta la distancia recorrida, si la distancia no rebasa los 300 Km., se cobra
una tarifa fija de S/.250, si la distancia recorrida es mayor a 300 km. y hasta
1000 km. se cobra la tarifa fija más el exceso de kilómetros a razón de S/.30
por km. y si la distancia recorrida es mayor a 1000 km. la compañía cobra la
tarifa fija más los kms. recorridos entre 300 hasta 1000 a razón de S/.30, más
S/.20 por km. de exceso en distancias mayores de 1000 km. Calcular el monto que
pagará un cliente.\\

%Pseudocódigo
\underline{\textit{Resolución en pseudocódigo}}\\ 
\inputminted
[
  frame=lines,
  framesep=2mm,
  baselinestretch=1.2,
  rulecolor=\color{black!30},
  %fontsize=\footnotesize,
  bgcolor=LightGray,
  %linenos
]
{./pseudocode.py:PseudocodeLexer -x}
{./pseudocodigo/007_practica.algo}

%C++
\underline{\textit{Resolución en C++}}\\
\inputminted
[
  frame=lines,
  framesep=2mm,
  baselinestretch=1.2,
  rulecolor=\color{black!30},
  %fontsize=\footnotesize,
  bgcolor=DarkGray,
  %linenos
]
{cpp}
{./cpp/007_practica.cpp}


\subsection*{Ejercicio 8}%
Una empresa registra el sexo, edad y estado civil de sus empleados a través de
un número entero positivo de cuatro cifras de acuerdo a lo siguiente: la
primera cifra de la izquierda representa el estado civil (1 para soltero, 2
para casado, 3 para viudo y 4 para divorciado), las siguientes dos cifras
representan la edad y la tercera cifra representa el sexo (1 para femenino y 2
para masculino). Diseñe un programa que determine el estado civil, edad y sexo
de un empleado conociendo el número que empaqueta dicha información.\\

%Pseudocódigo
\underline{\textit{Resolución en pseudocódigo}}\\ 
\inputminted
[
  frame=lines,
  framesep=2mm,
  baselinestretch=1.2,
  rulecolor=\color{black!30},
  %fontsize=\footnotesize,
  bgcolor=LightGray,
  %linenos
]
{./pseudocode.py:PseudocodeLexer -x}
{./pseudocodigo/008_practica.algo}

%C++
\underline{\textit{Resolución en C++}}\\
\inputminted
[
  frame=lines,
  framesep=2mm,
  baselinestretch=1.2,
  rulecolor=\color{black!30},
  %fontsize=\footnotesize,
  bgcolor=DarkGray,
  %linenos
]
{cpp}
{./cpp/008_practica.cpp}


\subsection*{Ejercicio 9}%
Calcular la comisión sobre las ventas totales de un empleado, sabiendo que el
empleado no recibe comisión si su venta es hasta S/.150, si la venta es
superior a S/.150 y menor o igual a S/.400 el empleado recibe una comisión del
10\% de las ventas y si las ventas son mayores a 400, entonces la comisión es
de S/.50 más el 9\% de las ventas.\\

%Pseudocódigo
\underline{\textit{Resolución en pseudocódigo}}\\ 
\inputminted
[
  frame=lines,
  framesep=2mm,
  baselinestretch=1.2,
  rulecolor=\color{black!30},
  %fontsize=\footnotesize,
  bgcolor=LightGray,
  %linenos
]
{./pseudocode.py:PseudocodeLexer -x}
{./pseudocodigo/009_practica.algo}

%C++
\underline{\textit{Resolución en C++}}\\
\inputminted
[
  frame=lines,
  framesep=2mm,
  baselinestretch=1.2,
  rulecolor=\color{black!30},
  %fontsize=\footnotesize,
  bgcolor=DarkGray,
  %linenos
]
{cpp}
{./cpp/009_practica.cpp}

\subsection*{Ejercicio 10}%
Dada la ecuación de la recta $y = mx + c$, y la ecuación de la circunferencia
$(x-a)^2+(y-b)^2 = r^2$, determinar los puntos de intersección de la recta con
la circunferencia, y analizar si la recta es secante o tangente a la
circunferencia.\\

%Pseudocódigo
\underline{\textit{Resolución en pseudocódigo}}\\ 
\inputminted
[
  frame=lines,
  framesep=2mm,
  baselinestretch=1.2,
  rulecolor=\color{black!30},
  %fontsize=\footnotesize,
  bgcolor=LightGray,
  %linenos
]
{./pseudocode.py:PseudocodeLexer -x}
{./pseudocodigo/010_practica.algo}


\subsection*{Ejercicio 11}%
Se necesita un sistema para un supermercado, en el cual si el monto de la
compra del cliente es mayor de \$5000 se le hará un descuento del 30\%, si es
menor o igual a \$5000 pero mayor que \$3000 será del 20\%, si no rebasa los
\$3000 pero si los \$1000 la rebaja efectiva es del 10\% y en caso de que no
rebase los \$1000 no tendrá beneficio.\\

%Pseudocódigo
\underline{\textit{Resolución en pseudocódigo}}\\ 
\inputminted
[
  frame=lines,
  framesep=2mm,
  baselinestretch=1.2,
  rulecolor=\color{black!30},
  %fontsize=\footnotesize,
  bgcolor=LightGray,
  %linenos
]
{./pseudocode.py:PseudocodeLexer -x}
{./pseudocodigo/011_practica.algo}

\subsection*{Ejercicio 12}%
Calcular la utilidad que un trabajador recibe en el reparto anual de utilidades
si este se le asigna como un porcentaje de su salario mensual que depende de su
antigüedad en la empresa de acuerdo con la siguiente tabla:\\

\begin{table}[htbp]
  \begin{center}
    \begin{tabular}{|l|l|}
      \hline
      Tiempo & Utilidad \\
      \hline \hline
      Menos de 1 año & 5\% del salario \\ \hline
      1 año o más y menos de 2 años & 7\% del salario \\ \hline
      2 años o más y menos de 5 años & 10\% del salario \\ \hline
      5 años o más y menos de 10 años & 15\% del salario \\ \hline
      10 años o más & 20\% del salario \\ \hline
    \end{tabular}
  \end{center}
\end{table}

%Pseudocódigo
\underline{\textit{Resolución en pseudocódigo}}\\ 
\inputminted
[
  frame=lines,
  framesep=2mm,
  baselinestretch=1.2,
  rulecolor=\color{black!30},
  %fontsize=\footnotesize,
  bgcolor=LightGray,
  %linenos
]
{./pseudocode.py:PseudocodeLexer -x}
{./pseudocodigo/012_practica.algo}

\subsection*{Ejercicio 13}%
Dado un número entero; determinar si el mismo es par, impar o nulo.\\

%Pseudocódigo
\underline{\textit{Resolución en pseudocódigo}}\\ 
\inputminted
[
  frame=lines,
  framesep=2mm,
  baselinestretch=1.2,
  rulecolor=\color{black!30},
  %fontsize=\footnotesize,
  bgcolor=LightGray,
  %linenos
]
{./pseudocode.py:PseudocodeLexer -x}
{./pseudocodigo/013_practica.algo}


\subsection*{Ejercicio 14}%
Escribir un programa que determine si un año es bisiesto. Un año es bisiesto si
es múltiplo de 4 (por ejemplo 1984). Los años múltiplos de 100 no son
bisiestos, salvo si ellos son también múltiplos de 400 (2000 es bisiesto, pero;
1800 no lo es).\\

%Pseudocódigo
\underline{\textit{Resolución en pseudocódigo}}\\ 
\inputminted
[
  frame=lines,
  framesep=2mm,
  baselinestretch=1.2,
  rulecolor=\color{black!30},
  %fontsize=\footnotesize,
  bgcolor=LightGray,
  %linenos
]
{./pseudocode.py:PseudocodeLexer -x}
{./pseudocodigo/014_practica.algo}

%C++
\underline{\textit{Resolución en C++}}\\
\inputminted
[
  frame=lines,
  framesep=2mm,
  baselinestretch=1.2,
  rulecolor=\color{black!30},
  %fontsize=\footnotesize,
  bgcolor=DarkGray,
  %linenos
]
{cpp}
{./cpp/014_practica.cpp}


\subsection*{Ejercicio 15}%
Elaborar un algoritmo en el que a partir de una fecha introducida por el
teclado con el formato Día, Mes, año, se obtenga la fecha del día siguiente.\\

%Pseudocódigo
\underline{\textit{Resolución en pseudocódigo}}\\ 
\inputminted
[
  frame=lines,
  framesep=2mm,
  baselinestretch=1.2,
  rulecolor=\color{black!30},
  %fontsize=\footnotesize,
  bgcolor=LightGray,
  %linenos
]
{./pseudocode.py:PseudocodeLexer -x}
{./pseudocodigo/015_practica.algo}

\subsection*{Ejercicio 16}%
La cantidad de días transcurridos entre dos fechas puede calcularse
transformándolas en días Julianos. Esta es una convención astronómica que
representa cada fecha como el número de días transcurridos desde el 1 de enero
de 4713 AC. Para transformar una fecha expresada como DIA, MES Y AÑO en días
Julianos se usa la siguiente fórmula:\\
$DJ = ENT(365.25*AP) + ENT(30.6001*MP) + DIA + 1720982$, donde DJ es el día
Juliano, y AP y MP son dos constantes que se obtiene como sigue:\\
Si MES = 1 ó 2: AP = ANO - 1 MP = MES + 13\\
Si MES $>$ 2: AP = AÑO MP = MES + 1\\
La cantidad de días entre dos fechas es igual a la diferencia entre los respectivos días Julianos:
dias = (dia Juliano2) - (dia Juliano 1)\\
Preparar un programa para ingresar las dos fechas como DIA1, MES1, AÑO1, Y
DIA2, MES2, AÑO2 respectivamente, y muestre la cantidad de días transcurridos
entre ambas.\\


%Pseudocódigo
\underline{\textit{Resolución en pseudocódigo}}\\ 
\inputminted
[
  frame=lines,
  framesep=2mm,
  baselinestretch=1.2,
  rulecolor=\color{black!30},
  %fontsize=\footnotesize,
  bgcolor=LightGray,
  %linenos
]
{./pseudocode.py:PseudocodeLexer -x}
{./pseudocodigo/016_practica.algo}

\subsection*{Ejercicio 17}%
Determinar la cantidad de dinero que recibirá un trabajador por concepto de las
horas extras trabajadas en une empresa, sabiendo que cuando las horas de
trabajo exceden de 40, el resto se consideran horas extras y que estas se pagan
al doble de una hora normal cuando no exceden de 8; si las horas extras exceden
de 8 se pagan las primeras 8 al doble de lo que se pagan las horas normales y
el resto al triple.\\

%Pseudocódigo
\underline{\textit{Resolución en pseudocódigo}}\\ 
\inputminted
[
  frame=lines,
  framesep=2mm,
  baselinestretch=1.2,
  rulecolor=\color{black!30},
  %fontsize=\footnotesize,
  bgcolor=LightGray,
  %linenos
]
{./pseudocode.py:PseudocodeLexer -x}
{./pseudocodigo/017_practica.algo}

\subsection*{Ejercicio 18}%
En una tienda de descuento se efectúa una promoción en la cual se hace un
descuento sobre el valor de la compra total según el color de la bolita que el
cliente saque al pagar en caja. Si la bolita es de color blanco no se le hará
descuento alguno, si es verde se le hará un 10\% de descuento, si es amarilla
un 25\%, si es azul un 50\% y si es roja un 100\%. Determinal la cantidad final
que el cliente deberá pagar por su compra. Se sabe que sólo hay bolitas de los
colores mencionados.\\

%Pseudocódigo
\underline{\textit{Resolución en pseudocódigo}}\\ 
\inputminted
[
  frame=lines,
  framesep=2mm,
  baselinestretch=1.2,
  rulecolor=\color{black!30},
  %fontsize=\footnotesize,
  bgcolor=LightGray,
  %linenos
]
{./pseudocode.py:PseudocodeLexer -x}
{./pseudocodigo/018_practica.algo}

\subsection*{Ejercicio 19}%
Ingrese seis notas y calcule el promedio, considerando las 5 mejores notas.\\

%Pseudocódigo
\underline{\textit{Resolución en pseudocódigo}}\\ 
\inputminted
[
  frame=lines,
  framesep=2mm,
  baselinestretch=1.2,
  rulecolor=\color{black!30},
  %fontsize=\footnotesize,
  bgcolor=LightGray,
  %linenos
]
{./pseudocode.py:PseudocodeLexer -x}
{./pseudocodigo/019_practica.algo}








%---------------------------------------------------------------------------------
\vspace{3cm} 
\section*{¡Envianos tus soluciones!}
Si estás llevando este curso con los profesores Cabrera, Romero o Salinas;
envíanos tus soluciones en los diferentes lenguajes de programación que
conozcas al correo \href{mailto:gprunecode@gmail.com}{gprunecode@gmail.com}.
n.n \\ 

Las mejores soluciones tanto en el algoritmo como en el código, serán
publicadas en las siguientes ediciones de estos documentos.\\ \\

El asunto del correo debe estar de la siguiente manera:\\
$NumeroDeSesion-Profesor-NumeroDeEjercicio$ \\ \\
Por ejemplo:  \\
$01-Romero-01$ \\

Y dentro del correo adjuntar tu solución y nombre como quieras ser reconocido en caso de ser electo.

\vspace{2cm}
\LARGE\textit{RuneCode}


\end{document}
