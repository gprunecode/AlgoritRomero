%%
%% (
%%  )\ )                             (
%%  (()/(   (            (             )\  )   (
%%   /(_))  ))\   (       ))\  (   (   (()/(   ))\
%%   (_))  /((_)  )\  )  /((_) )\  )\   ((_))/((_)
%%   | _ \(_))(  _(_/( (_) )  ((_)((_)  _| |(_))
%%   |   /| || || ' \))/ -_)/ _|/ _ \/ _` |/ -_)
%%   |_|_\ \_,_||_||_| \___|\__|\___/\__,_|\___|
%%

\documentclass{article}
\usepackage[utf8x]{inputenc}
\usepackage{amsmath}
\usepackage{amsfonts}
\usepackage{amssymb}
\usepackage{graphicx} % Paquete para incluir imágenes en el documento LaTeX
\usepackage{hyperref}
\hypersetup{
  colorlinks=true,
  linkcolor=blue,
  filecolor=magenta,
  urlcolor=cyan,
}
\urlstyle{same}
\usepackage{varwidth}

\newcommand\tab[1][1cm]{\hspace*{#1}}

\usepackage{multirow}

\usepackage[a4paper,rmargin=1.5cm,lmargin=1.5cm,top=1.5cm,bottom=1.5cm]{geometry}

\usepackage{pdfpages}

\usepackage{xcolor}
\usepackage{minted}
\usemintedstyle[cpp]{monokai}
\usemintedstyle[python]{paraiso-dark}
\usemintedstyle[./pseudocode.py:PseudocodeLexer -x]{rainbow_dash}
\definecolor{LightGray}{gray}{0.98}
\definecolor{DarkGray}{gray}{0.1}
\definecolor{MidGray}{gray}{0.8}
\definecolor{codegreen}{rgb}{0,0.6,0}
\definecolor{codegray}{rgb}{0.5,0.5,0.5}
\definecolor{codepurple}{rgb}{0.58,0,0.82}
\definecolor{backcolour}{rgb}{0.95,0.95,0.92}

\setlength{\parindent}{0px}  % Setea la indentacion de la primera linea de cada parrafo a cero pixeles.


\title{Resolución de la primera semana}
\author{@RuneCode}

\begin{document}
%% Portada
\includepdf{./portada/portada.pdf}


%% ####################################################################################
%%    Inicio del Documento
%% ####################################################################################
\section*{Segunda Semana}%
En esta clase se revisaron los temas de:
\begin{itemize}
\item \textbf{Condicional Simple}
\item \textbf{Condicional Doble}
\item \textbf{Condicional anidado}
\item \textbf{Condicional múltiple}
\item \textbf{Uso de operadores lógicos}
\end{itemize}
\vspace{1cm}
\textbf{Nota:} Te recomendamos realizar los ejercicios antes de ver su solución
ya que el objetivo de este documento es presentarte una de las muchas posibles
soluciones y así tengas una modelo con el que puedas comparar. Si tienes alguna
sugerencia de mejora puedes comunicarte por correo a la dirección
\href{mailto:gprunecode@gmail.com}{gprunecode@gmail.com}.

%% ####################################################################################
%%    Condicional Simple
%% ####################################################################################
\section{Condicional Simple}%
Ejecuta las sentencias en el caso que la expresion lógica sea Verdad.

%Pseudocódigo
\inputminted
[
  frame=lines,
  framesep=2mm,
  baselinestretch=1.2,
  rulecolor=\color{black!30},
  %fontsize=\footnotesize,
  bgcolor=LightGray,
  %linenos
]
{./pseudocode.py:PseudocodeLexer -x}
{./pseudocodigo/condicional_simple.algo}


\subsection*{Ejemplo 1}%
Los trabajadores con más de 10 años de antigüedad recibirán un bono de 10\% de
su básico. Hallar bono y total ingresos.\\

%Pseudocódigo
\underline{\textit{Resolución en pseudocódigo}}\\ 
\inputminted
[
  frame=lines,
  framesep=2mm,
  baselinestretch=1.2,
  rulecolor=\color{black!30},
  %fontsize=\footnotesize,
  bgcolor=LightGray,
  %linenos
]
{./pseudocode.py:PseudocodeLexer -x}
{./pseudocodigo/001_ejemplo.algo}
\clearpage

%C++
\underline{\textit{Resolución en C++}}\\
\inputminted
[
  frame=lines,
  framesep=2mm,
  baselinestretch=1.2,
  rulecolor=\color{black!30},
  %fontsize=\footnotesize,
  bgcolor=DarkGray,
  %linenos
]
{cpp}
{./cpp/001_ejemplo.cpp}


%% ####################################################################################
%%    Condicional Doble
%% ####################################################################################
\section{Condicional Doble}%
Ejecuta sentenciaV si expresionLogica es Verda y ejecutará setenciaF si
expresionLogica es Falsa.

%Pseudocódigo
\inputminted
[
  frame=lines,
  framesep=2mm,
  baselinestretch=1.2,
  rulecolor=\color{black!30},
  %fontsize=\footnotesize,
  bgcolor=LightGray,
  %linenos
]
{./pseudocode.py:PseudocodeLexer -x}
{./pseudocodigo/condicional_doble.algo}


\subsection*{Ejemplo 2}%
Los trabajadores con más de 10 años de antigüedad recibirán un bono de 10\% de
su básico, los demás sólo 5\%. Hallar bono y total de ingresos.\\

%Pseudocódigo
\underline{\textit{Resolución en pseudocódigo}}\\ 
\inputminted
[
  frame=lines,
  framesep=2mm,
  baselinestretch=1.2,
  rulecolor=\color{black!30},
  %fontsize=\footnotesize,
  bgcolor=LightGray,
  %linenos
]
{./pseudocode.py:PseudocodeLexer -x}
{./pseudocodigo/002_ejemplo.algo}

%C++
\underline{\textit{Resolución en C++}}\\
\inputminted
[
  frame=lines,
  framesep=2mm,
  baselinestretch=1.2,
  rulecolor=\color{black!30},
  %fontsize=\footnotesize,
  bgcolor=DarkGray,
  %linenos
]
{cpp}
{./cpp/002_ejemplo.cpp}


%% ####################################################################################
%%    Condicional Anidado
%% ####################################################################################
\section{Condicional Anidado}%
Se ubica un condicional simple o doble dentro de otro condicional. Se puede
anidar en la parte entonces o en sino. Se sugiere ubicar el anidamiento en la
sección sino.

%Pseudocódigo
\inputminted
[
  frame=lines,
  framesep=2mm,
  baselinestretch=1.2,
  rulecolor=\color{black!30},
  %fontsize=\footnotesize,
  bgcolor=LightGray,
  %linenos
]
{./pseudocode.py:PseudocodeLexer -x}
{./pseudocodigo/condicional_anidado.algo}


\subsection*{Ejemplo 3}%
La empresa "Sedapal" facturará para consumos inferiores a 100 $m^3$, S/.1 x
$m^3$ para consumos de 100 hasta 500 $m^3$, S/.1.5 x $m^3$ y para consumos
superiores a 500 $m^3$, S/.2 x $m^3$. Hallar importe a pagar.\\

%Pseudocódigo
\underline{\textit{Resolución en pseudocódigo}}\\ 
\inputminted
[
  frame=lines,
  framesep=2mm,
  baselinestretch=1.2,
  rulecolor=\color{black!30},
  %fontsize=\footnotesize,
  bgcolor=LightGray,
  %linenos
]
{./pseudocode.py:PseudocodeLexer -x}
{./pseudocodigo/003_ejemplo.algo}

%C++
\underline{\textit{Resolución en C++}}\\
\inputminted
[
  frame=lines,
  framesep=2mm,
  baselinestretch=1.2,
  rulecolor=\color{black!30},
  %fontsize=\footnotesize,
  bgcolor=DarkGray,
  %linenos
]
{cpp}
{./cpp/003_ejemplo.cpp}


%% ####################################################################################
%%    Condicional Multiple
%% ####################################################################################
\section{Condicional Múltiple}%
Usaremos caso ... vale. Sólo una de las opciones será verdadera y se ejecutarán
las sentencias asociadas a ella. La variable evaluada debe ser entera o
carácter.

%Pseudocódigo
\inputminted
[
  frame=lines,
  framesep=2mm,
  baselinestretch=1.2,
  rulecolor=\color{black!30},
  %fontsize=\footnotesize,
  bgcolor=LightGray,
  %linenos
]
{./pseudocode.py:PseudocodeLexer -x}
{./pseudocodigo/condicional_multiple.algo}

\subsection*{Ejemplo 4}%
Ingresar un entero (de 1 a 7) y mostrar el día correspondiente.\\

%Pseudocódigo
\underline{\textit{Resolución en pseudocódigo}}\\ 
\inputminted
[
  frame=lines,
  framesep=2mm,
  baselinestretch=1.2,
  rulecolor=\color{black!30},
  %fontsize=\footnotesize,
  bgcolor=LightGray,
  %linenos
]
{./pseudocode.py:PseudocodeLexer -x}
{./pseudocodigo/004_ejemplo.algo}

%C++
\underline{\textit{Resolución en C++}}\\
\inputminted
[
  frame=lines,
  framesep=2mm,
  baselinestretch=1.2,
  rulecolor=\color{black!30},
  %fontsize=\footnotesize,
  bgcolor=DarkGray,
  %linenos
]
{cpp}
{./cpp/004_ejemplo.cpp}


%% ####################################################################################
%%    Operadores logicos
%% ####################################################################################
\section{Uso de operadores lógicos}%
\begin{itemize}
  \item El resultado de una expresión lógica es el valor verdadero o falso.
  \item Operadores relacionales: \textbf{$=$, $<>$, $<$, $<=$, $>$, $>=$}
  \item Operadores Lógicos: \textbf{No, Y, O}
  \item Para plantear expresiones lógicas más complejas se puede usar operadores lógicos.
\end{itemize}

\subsection*{Ejercicio 1}%
Calcular las raíces de una ecuación de segundo grado. Considere las diferentes
situaciones que se pueden dar.

\begin{equation*}
  x = \frac{-b \pm \sqrt{b^2 - 4ac}}{2a}
\end{equation*}

%Pseudocódigo
\textit{Resolución en pseudocódigo}\\ 
\inputminted
[
  frame=lines,
  framesep=2mm,
  baselinestretch=1.2,
  rulecolor=\color{black!30},
  %fontsize=\footnotesize,
  bgcolor=LightGray,
  %linenos
]
{./pseudocode.py:PseudocodeLexer -x}
{./pseudocodigo/001_ejercicio.algo}

%C++
\textit{Resolución en C++}\\
\inputminted
[
  frame=lines,
  framesep=2mm,
  baselinestretch=1.2,
  rulecolor=\color{black!30},
  %fontsize=\footnotesize,
  bgcolor=DarkGray,
  %linenos
]
{cpp}
{./cpp/001_ejercicio.cpp}

\clearpage

\subsection*{Ejercicio 2}%
Calcular el pago por ciclo de un alumno de una Universidad, si se ingresan,
créditos inscritos, categoría, matrícula (1: normal, 2: extemporánea). El pago
por crédito depende de la categoría de acuerdo a la siguiente tabla:

\begin{table}[htbp]
  \begin{center}
    \begin{tabular}{|l|l|}
      \hline
      CATEGORÍA & PAGO CRÉDITO \\
      \hline \hline
      A & 125.00 \\ \hline
      B & 150.00 \\ \hline
      C & 180.00 \\ \hline
    \end{tabular}
  \end{center}
\end{table}
Por matrícula extemporánea se paga un recargo de 40.00 soles.\\

%Pseudocódigo
\textit{Resolución en pseudocódigo}\\ 
\inputminted
[
  frame=lines,
  framesep=2mm,
  baselinestretch=1.2,
  rulecolor=\color{black!30},
  %fontsize=\footnotesize,
  bgcolor=LightGray,
  %linenos
]
{./pseudocode.py:PseudocodeLexer -x}
{./pseudocodigo/002_ejercicio.algo}

%C++
\textit{Resolución en C++}\\
\inputminted
[
  frame=lines,
  framesep=2mm,
  baselinestretch=1.2,
  rulecolor=\color{black!30},
  %fontsize=\footnotesize,
  bgcolor=DarkGray,
  %linenos
]
{cpp}
{./cpp/002_ejercicio.cpp}



%---------------------------------------------------------------------------------
\vspace{3cm} 
\section*{¡Envianos tus soluciones!}
Si estás llevando este curso con los profesores Cabrera, Romero o Salinas;
envíanos tus soluciones en los diferentes lenguajes de programación que
conozcas al correo \href{mailto:gprunecode@gmail.com}{gprunecode@gmail.com}.
n.n \\ 

Las mejores soluciones tanto en el algoritmo como en el código, serán
publicadas en las siguientes ediciones de estos documentos.\\ \\

El asunto del correo debe estar de la siguiente manera:\\
$NumeroDeSesion-Profesor-NumeroDeEjercicio$ \\ \\
Por ejemplo:  \\
$01-Romero-01$ \\

Y dentro del correo adjuntar tu solución y nombre como quieras ser reconocido en caso de ser electo.

\vspace{2cm}
\LARGE\textit{RuneCode}


\end{document}
