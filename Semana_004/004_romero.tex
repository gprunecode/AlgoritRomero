%%
%% (
%%  )\ )                             (
%%  (()/(   (            (             )\  )   (
%%   /(_))  ))\   (       ))\  (   (   (()/(   ))\
%%   (_))  /((_)  )\  )  /((_) )\  )\   ((_))/((_)
%%   | _ \(_))(  _(_/( (_) )  ((_)((_)  _| |(_))
%%   |   /| || || ' \))/ -_)/ _|/ _ \/ _` |/ -_)
%%   |_|_\ \_,_||_||_| \___|\__|\___/\__,_|\___|
%%

\documentclass{article}
\usepackage[utf8x]{inputenc}
\usepackage{amsmath}
%\usepackage{slashbox}
\usepackage{amsfonts}
\usepackage{amssymb}
\usepackage{graphicx} % Paquete para incluir imágenes en el documento LaTeX
\usepackage{hyperref}
\hypersetup{
  colorlinks=true,
  linkcolor=blue,
  filecolor=magenta,
  urlcolor=cyan,
}
\urlstyle{same}
\usepackage{varwidth}

\newcommand\tab[1][1cm]{\hspace*{#1}}

\usepackage{multirow}

\usepackage[a4paper,rmargin=1.5cm,lmargin=1.5cm,top=1.5cm,bottom=1.5cm]{geometry}

\usepackage{pdfpages}

\usepackage{xcolor}
\usepackage{minted}
\setminted[cpp]{frame=lines, framesep=2mm, baselinestretch=1.2, rulecolor=\color{black!80}, bgcolor=DarkGray}
\usemintedstyle[cpp]{monokai}
\setminted[python3]{frame=lines, framesep=2mm, baselinestretch=1.2, rulecolor=\color{black!80}, bgcolor=DarkGray}
\usemintedstyle[python]{paraiso-dark}
\setminted[./pseudocode.py:PseudocodeLexer -x]{frame=lines, framesep=2mm, baselinestretch=1.2,
            rulecolor=\color{black!30}, bgcolor=LightGray}
\usemintedstyle[./pseudocode.py:PseudocodeLexer -x]{rainbow_dash}
\setminted[bash]{baselinestretch=1.2,rulecolor=\color{black!30},fontsize=\footnotesize,bgcolor=LightGray}
\definecolor{LightGray}{gray}{0.98}
\definecolor{DarkGray}{gray}{0.1}
\definecolor{MidGray}{gray}{0.8}
\definecolor{codegreen}{rgb}{0,0.6,0}
\definecolor{codegray}{rgb}{0.5,0.5,0.5}
\definecolor{codepurple}{rgb}{0.58,0,0.82}
\definecolor{backcolour}{rgb}{0.95,0.95,0.92}

\setlength{\parindent}{0px}  % Setea la indentacion de la primera linea de cada parrafo a cero pixeles.


\title{Resolución de la primera semana}
\author{@RuneCode}

\begin{document}
%% Portada
%\includepdf{./portada/portada.pdf}

\subsection*{Ejercicio 31}%

%pseudocodigo
\underline{\textit{Resolución en pseudocodigo}}\\
\inputminted{./pseudocode.py:PseudocodeLexer -x}{./pseudocodigo/031_pseudo.algo}



%\section*{Agradecimientos}
%\textbf{Personas que apoyaron mandando psudocódigo y código en este documento:}\\
%
%%---------------------------------------------------------------------------------
%\vspace{3cm} 
%\section*{¡Envianos tus soluciones!}
%Si estás llevando este curso con los profesores Cabrera, Romero o Salinas;
%envíanos tus soluciones en los diferentes lenguajes de programación que
%conozcas al correo \href{mailto:gprunecode@gmail.com}{gprunecode@gmail.com}.
%n.n \\ 
%
%Las mejores soluciones tanto en el algoritmo como en el código, serán
%publicadas en las siguientes ediciones de estos documentos.\\
%
%El asunto del correo debe estar de la siguiente manera:\\
%$NumeroDeSesion-Profesor-NumeroDeEjercicio$ \\
%Por ejemplo:  \\
%$01-Romero-01$ \\
%
%Y dentro del correo adjuntar tu solución y nombre como quieras ser reconocido en caso de ser electo.
%
%\vspace{2cm}
%\LARGE\textit{RuneCode}
%

\end{document}

