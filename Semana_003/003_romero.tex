%%
%% (
%%  )\ )                             (
%%  (()/(   (            (             )\  )   (
%%   /(_))  ))\   (       ))\  (   (   (()/(   ))\
%%   (_))  /((_)  )\  )  /((_) )\  )\   ((_))/((_)
%%   | _ \(_))(  _(_/( (_) )  ((_)((_)  _| |(_))
%%   |   /| || || ' \))/ -_)/ _|/ _ \/ _` |/ -_)
%%   |_|_\ \_,_||_||_| \___|\__|\___/\__,_|\___|
%%

\documentclass{article}
\usepackage[utf8x]{inputenc}
\usepackage{amsmath}
\usepackage{diagbox}
\usepackage{amsfonts}
\usepackage{amssymb}
\usepackage{graphicx} % Paquete para incluir imágenes en el documento LaTeX
\usepackage{hyperref}
\hypersetup{
  colorlinks=true,
  linkcolor=blue,
  filecolor=magenta,
  urlcolor=cyan,
}
\urlstyle{same}
\usepackage{varwidth}

\newcommand\tab[1][1cm]{\hspace*{#1}}

\usepackage{multirow}

\usepackage[a4paper,rmargin=1.5cm,lmargin=1.5cm,top=1.5cm,bottom=1.5cm]{geometry}

\usepackage{pdfpages}

\usepackage{xcolor}
\usepackage{minted}
\usemintedstyle[cpp]{monokai}
\usemintedstyle[python]{paraiso-dark}
\usemintedstyle[./pseudocode.py:PseudocodeLexer -x]{rainbow_dash}
\definecolor{LightGray}{gray}{0.98}
\definecolor{DarkGray}{gray}{0.1}
\definecolor{MidGray}{gray}{0.8}
\definecolor{codegreen}{rgb}{0,0.6,0}
\definecolor{codegray}{rgb}{0.5,0.5,0.5}
\definecolor{codepurple}{rgb}{0.58,0,0.82}
\definecolor{backcolour}{rgb}{0.95,0.95,0.92}

\setlength{\parindent}{0px}  % Setea la indentacion de la primera linea de cada parrafo a cero pixeles.


\title{Resolución de la primera semana}
\author{@RuneCode}

\begin{document}
%% Portada
\includepdf{./portada/portada.pdf}


%% ####################################################################################
%%    Inicio del Documento
%% ####################################################################################
\section*{Tercera Semana}%
En esta clase se revisaron los temas de:
\begin{itemize}
\item \textbf{Iteraciones}
\item \textbf{Mientras hacer}
\item \textbf{Contador y acumulador}
\item \textbf{Hacer Mientras}
\item \textbf{Para hacer}
\end{itemize}
\vspace{1cm}
\textbf{Nota:} Te recomendamos realizar los ejercicios antes de ver su solución
ya que el objetivo de este documento es presentarte una de las muchas posibles
soluciones y así tengas una modelo con el que puedas comparar. Si tienes alguna
sugerencia de mejora puedes comunicarte por correo a la dirección
\href{mailto:gprunecode@gmail.com}{gprunecode@gmail.com}.

%% ####################################################################################
%%    Iteraciones o Repetitivas
%% ####################################################################################
\section*{Iteraciones o Repetitivas}%
Existen soluciones en las que es necesario repetir el mismo conjunto de
aciones, repetidas veces.\\
Lo anterior sólo es posible si reescribimos las acciones cuantas veces sea
necesario que se ejecuten, lo que puede ser una tarea bastante tediosa.\\
Presentaremos un conjunto de acciones conocidas con el nombre de Iterativas,
cuya finalidad es permitir que un determinado conjunto de acciones se repitan
un número requerido de veces.

\section{Iteraciones: Mientras}%

%Pseudocódigo
\inputminted
[
  frame=lines,
  framesep=2mm,
  baselinestretch=1.2,
  rulecolor=\color{black!30},
  %fontsize=\footnotesize,
  bgcolor=LightGray,
  %linenos
]
{./pseudocode.py:PseudocodeLexer -x}
{./pseudocodigo/001_mientras.algo}

En esta estructura de control repetitiva, se evalúa la expresion booleana, si
es verdadera se ejecuta la secuencia de acciones.\\
Mientras la expresion booleana sea verdadera seguirá ejecutándose la secuencia
de acciones.\\
Cuando la expresión booleana se haga falsa se finalizará la iteración.\\

\subsection*{Ejemplo 1}%
Escriba el pseudocódigo para hallar la suma de los n primeros términos de la siguiente serie:\\
\begin{equation*}
  1 + 2 + 3 + 4 + 5 + \cdots
\end{equation*}

\vspace{0.7cm}

%Pseudocódigo
\underline{\textit{Resolución en pseudocódigo}}\\ 
\inputminted
[
  frame=lines,
  framesep=2mm,
  baselinestretch=1.2,
  rulecolor=\color{black!30},
  %fontsize=\footnotesize,
  bgcolor=LightGray,
  %linenos
]
{./pseudocode.py:PseudocodeLexer -x}
{./pseudocodigo/001_ejemplomh.algo}

%C++
\underline{\textit{Resolución en C++}}\\
\inputminted
[
  frame=lines,
  framesep=2mm,
  baselinestretch=1.2,
  rulecolor=\color{black!30},
  %fontsize=\footnotesize,
  bgcolor=DarkGray,
  %linenos
]
{cpp}
{./cpp/001_ejemplomh.cpp}


\section{Contador y Acumulador}%
Un contador es un elemento que usamos con las estructuras repetitivas y que se
incrementa en un valor fijo en cada iteración.\\

%Pseudocódigo
\inputminted
[
  frame=lines,
  framesep=2mm,
  baselinestretch=1.2,
  rulecolor=\color{black!30},
  %fontsize=\footnotesize,
  bgcolor=LightGray,
  %linenos
]
{./pseudocode.py:PseudocodeLexer -x}
{./pseudocodigo/contador.algo}

Un acumulador a diferencia del contador se incrementa en un valor diferente en
cada iteración. El resultado que se obtiene es la suma total de todos los
valores.

%Pseudocódigo
\inputminted
[
  frame=lines,
  framesep=2mm,
  baselinestretch=1.2,
  rulecolor=\color{black!30},
  %fontsize=\footnotesize,
  bgcolor=LightGray,
  %linenos
]
{./pseudocode.py:PseudocodeLexer -x}
{./pseudocodigo/acumulador.algo}
\newpage

\subsection*{Ejercicio 1}%
Calcule el valor de la siguiente serie para N términos:\\
\begin{equation*}
  \dfrac{1}{2} + \dfrac{2}{3} + \dfrac{3}{4} + \cdots
\end{equation*}

%Pseudocódigo
\underline{\textit{Resolución en pseudocódigo}}\\ 
\inputminted
[
  frame=lines,
  framesep=2mm,
  baselinestretch=1.2,
  rulecolor=\color{black!30},
  %fontsize=\footnotesize,
  bgcolor=LightGray,
  %linenos
]
{./pseudocode.py:PseudocodeLexer -x}
{./pseudocodigo/001_ejercicio.algo}

%C++
\underline{\textit{Resolución en C++}}\\
\inputminted
[
  frame=lines,
  framesep=2mm,
  baselinestretch=1.2,
  rulecolor=\color{black!30},
  %fontsize=\footnotesize,
  bgcolor=DarkGray,
  %linenos
]
{cpp}
{./cpp/001_ejerciciomh.cpp}
\newpage

\subsection*{Ejemplo 2}%
Escribir el pseudocódigo para hallar el promedio de un alumno que tiene n notas.\\

%Pseudocódigo
\underline{\textit{Resolución en pseudocódigo}}\\ 
\inputminted
[
  frame=lines,
  framesep=2mm,
  baselinestretch=1.2,
  rulecolor=\color{black!30},
  %fontsize=\footnotesize,
  bgcolor=LightGray,
  %linenos
]
{./pseudocode.py:PseudocodeLexer -x}
{./pseudocodigo/002_ejemplomh.algo}

%C++
\underline{\textit{Resolución en C++}}\\
\inputminted
[
  frame=lines,
  framesep=2mm,
  baselinestretch=1.2,
  rulecolor=\color{black!30},
  %fontsize=\footnotesize,
  bgcolor=DarkGray,
  %linenos
]
{cpp}
{./cpp/002_ejemplomh.cpp}


\section{Iteraciones: Hacer ... Mientras}%

%Pseudocódigo
\inputminted
[
  frame=lines,
  framesep=2mm,
  baselinestretch=1.2,
  rulecolor=\color{black!30},
  %fontsize=\footnotesize,
  bgcolor=LightGray,
  %linenos
]
{./pseudocode.py:PseudocodeLexer -x}
{./pseudocodigo/002_hacermientras.algo}

La secuencia de n acciones se ejecuta y luego se evalúa la ExpresiónBooleana,
si es verdadera, nuevamente ejecutará la secuencia de n acciones.\\
Esto continuará mientras que la ExpresiónBooleana sea verdadera.\\
Cuando la ExpresiónBooleana se haga falsa, finalizará la ejecución de la
estructura repetitiva.\\

\subsection*{Ejemplo 2}%
Hallar el promedio de cada uno de n alumnos.\\
Considera que cada alumno tiene 3 notas: examen parcial, examen final y práctica.

%Pseudocódigo
\underline{\textit{Resolución en pseudocódigo}}\\ 
\inputminted
[
  frame=lines,
  framesep=2mm,
  baselinestretch=1.2,
  rulecolor=\color{black!30},
  %fontsize=\footnotesize,
  bgcolor=LightGray,
  %linenos
]
{./pseudocode.py:PseudocodeLexer -x}
{./pseudocodigo/002_ejemplohm.algo}

%C++
\underline{\textit{Resolución en C++}}\\
\inputminted
[
  frame=lines,
  framesep=2mm,
  baselinestretch=1.2,
  rulecolor=\color{black!30},
  %fontsize=\footnotesize,
  bgcolor=DarkGray,
  %linenos
]
{cpp}
{./cpp/002_ejemplohm.cpp}

\subsection*{Problema}%
En una organización hay n trabajadores. Cada trabajador tiene un sueldo. Se pide hallar:\\
\begin{enumerate}
  \item ¿Cuánto paga en total la organización a sus trabajadores?
  \item ¿Cuántos trabajadores ganan más de 5000 soles?
\end{enumerate}

%Pseudocódigo
\underline{\textit{Resolución en pseudocódigo}}\\ 
\inputminted
[
  frame=lines,
  framesep=2mm,
  baselinestretch=1.2,
  rulecolor=\color{black!30},
  %fontsize=\footnotesize,
  bgcolor=LightGray,
  %linenos
]
{./pseudocode.py:PseudocodeLexer -x}
{./pseudocodigo/001_ejerciciohm.algo}

%C++
\underline{\textit{Resolución en C++}}\\
\inputminted
[
  frame=lines,
  framesep=2mm,
  baselinestretch=1.2,
  rulecolor=\color{black!30},
  %fontsize=\footnotesize,
  bgcolor=DarkGray,
  %linenos
]
{cpp}
{./cpp/002_ejerciciohm.cpp}


\subsection*{Ejercicio 2}%
Suponga que actualmente las poblaciones de los países A y B son 52 y 85
millones de habitantes respectivamente. Supongamos que las tasas de crecimiento
de población son 6\% y 4\% respectivamente. Determine en qué año la población A
excede a la población B.\\
Resuelva el problema considerando que todos los valores son datos. Realice las
verificaciones necesarias.

%Pseudocódigo
\underline{\textit{Resolución en pseudocódigo}}\\ 
\inputminted
[
  frame=lines,
  framesep=2mm,
  baselinestretch=1.2,
  rulecolor=\color{black!30},
  %fontsize=\footnotesize,
  bgcolor=LightGray,
  %linenos
]
{./pseudocode.py:PseudocodeLexer -x}
{./pseudocodigo/002_ejerciciohm.algo}

%C++
\underline{\textit{Resolución en C++}}\\
\inputminted
[
  frame=lines,
  framesep=2mm,
  baselinestretch=1.2,
  rulecolor=\color{black!30},
  %fontsize=\footnotesize,
  bgcolor=DarkGray,
  %linenos
]
{cpp}
{./cpp/002_ejerciciohm2.cpp}


\section{Iteraciones: Para}%

%Pseudocódigo
\inputminted
[
  frame=lines,
  framesep=2mm,
  baselinestretch=1.2,
  rulecolor=\color{black!30},
  %fontsize=\footnotesize,
  bgcolor=LightGray,
  %linenos
]
{./pseudocode.py:PseudocodeLexer -x}
{./pseudocodigo/003_para.algo}

La \textbf{Variable} se inicializa con valor \textit{Inicial} y si ésta es
menor o igual a \textit{valor Final} entonces ejecuta la secuencia de n
acciones.\\
Luego de ejecutar las acciones la \textbf{Variable} se incrementa en valor
\textit{valor Incremento} y vuelve a compararse con \textit{valor Final}.\\
Cuando la \textbf{Variable} excede el \textit{valor Final}, se finalizan las
iteraciones.
\vspace{1cm}

\subsection*{Ejemplo 3}%
Calcular la suma de n términos de la serie.
\begin{equation*}
  1 + 2 + 3 + 3 + 4 + 5 + \cdots + n
\end{equation*}
\newpage

%Pseudocódigo
\underline{\textit{Resolución en pseudocódigo}}\\ 
\inputminted
[
  frame=lines,
  framesep=2mm,
  baselinestretch=1.2,
  rulecolor=\color{black!30},
  %fontsize=\footnotesize,
  bgcolor=LightGray,
  %linenos
]
{./pseudocode.py:PseudocodeLexer -x}
{./pseudocodigo/003_ejemplop.algo}

%C++
\underline{\textit{Resolución en C++}}\\
\inputminted
[
  frame=lines,
  framesep=2mm,
  baselinestretch=1.2,
  rulecolor=\color{black!30},
  %fontsize=\footnotesize,
  bgcolor=DarkGray,
  %linenos
]
{cpp}
{./cpp/003_ejemplop.cpp}


\subsection*{Ejemplo 4}%
Hallar el producto de los n primeros números naturales.\\
\begin{equation*}
  1 * 2 * 3 * 4 * \cdots
\end{equation*}

%Pseudocódigo
\underline{\textit{Resolución en pseudocódigo}}\\ 
\inputminted
[
  frame=lines,
  framesep=2mm,
  baselinestretch=1.2,
  rulecolor=\color{black!30},
  %fontsize=\footnotesize,
  bgcolor=LightGray,
  %linenos
]
{./pseudocode.py:PseudocodeLexer -x}
{./pseudocodigo/004_ejemplop.algo}

%C++
\underline{\textit{Resolución en C++}}\\
\inputminted
[
  frame=lines,
  framesep=2mm,
  baselinestretch=1.2,
  rulecolor=\color{black!30},
  %fontsize=\footnotesize,
  bgcolor=DarkGray,
  %linenos
]
{cpp}
{./cpp/004_ejemplop.cpp}


\subsection*{Ejercicios}%
Escriba un pseudocódigo que permita multiplicar dos números entereos positivos
usando solamente la operación suma.\\

%Pseudocódigo
\underline{\textit{Resolución en pseudocódigo}}\\ 
\inputminted
[
  frame=lines,
  framesep=2mm,
  baselinestretch=1.2,
  rulecolor=\color{black!30},
  %fontsize=\footnotesize,
  bgcolor=LightGray,
  %linenos
]
{./pseudocode.py:PseudocodeLexer -x}
{./pseudocodigo/003_ejerciciop.algo}

%C++
\underline{\textit{Resolución en C++}}\\
\inputminted
[
  frame=lines,
  framesep=2mm,
  baselinestretch=1.2,
  rulecolor=\color{black!30},
  %fontsize=\footnotesize,
  bgcolor=DarkGray,
  %linenos
]
{cpp}
{./cpp/005_ejemplo.cpp}


\subsection*{Ejercicios}%
Determinar la cantidad de dígitos que tiene un número entero y además mostrar
la suma de los dígitos pares y de los impares. Considere el cero como número
par.\\

%Pseudocódigo
\underline{\textit{Resolución en pseudocódigo}}\\ 
\inputminted
[
  frame=lines,
  framesep=2mm,
  baselinestretch=1.2,
  rulecolor=\color{black!30},
  %fontsize=\footnotesize,
  bgcolor=LightGray,
  %linenos
]
{./pseudocode.py:PseudocodeLexer -x}
{./pseudocodigo/004_ejerciciop.algo}

%C++
\underline{\textit{Resolución en C++}}\\
\inputminted
[
  frame=lines,
  framesep=2mm,
  baselinestretch=1.2,
  rulecolor=\color{black!30},
  %fontsize=\footnotesize,
  bgcolor=DarkGray,
  %linenos
]
{cpp}
{./cpp/006_ejemplo.cpp}

\newpage
\section*{Agradecimientos}
\textbf{Personas que apoyaron mandando psudocódigo y código en este documento:}\\

%---------------------------------------------------------------------------------
\vspace{3cm} 
\section*{¡Envianos tus soluciones!}
Si estás llevando este curso con los profesores Cabrera, Romero o Salinas;
envíanos tus soluciones en los diferentes lenguajes de programación que
conozcas al correo \href{mailto:gprunecode@gmail.com}{gprunecode@gmail.com}.
n.n \\ 

Las mejores soluciones tanto en el algoritmo como en el código, serán
publicadas en las siguientes ediciones de estos documentos.\\ \\

El asunto del correo debe estar de la siguiente manera:\\
$NumeroDeSesion-Profesor-NumeroDeEjercicio$ \\ \\
Por ejemplo:  \\
$01-Romero-01$ \\

Y dentro del correo adjuntar tu solución y nombre como quieras ser reconocido en caso de ser electo.

\vspace{2cm}
\LARGE\textit{RuneCode}


\end{document}

